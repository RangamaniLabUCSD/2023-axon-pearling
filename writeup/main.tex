\documentclass[11pt]{article}
%---------------------------------
% Set page dimensions
%---------------------------------
\usepackage[margin=1in]{geometry}
% \setlength{\parindent}{0em}
\setlength{\parskip}{0em}
%=================================

% \usepackage[T1]{fontenc}
% \usepackage[utf8]{inputenc}
% \usepackage{times}
% \usepackage{helvet}

%---------------------------------
% Font options
%---------------------------------
\usepackage{fontspec}
\setmainfont{Arial}

\usepackage[style=science, doi=true, backend=biber]{biblatex}
\setlength{\bibitemsep}{1pt}
% \renewcommand\bibfont{\normalfont\scriptsize}
\addbibresource{ref.bib}

\usepackage{graphicx}
\usepackage{float}
\PassOptionsToPackage{hyphens}{url}\usepackage[colorlinks=false]{hyperref}
\usepackage{bookmark}
\usepackage[all]{hypcap}
\usepackage[version=4]{mhchem}
\usepackage[capitalise]{cleveref}

\usepackage{wrapfig}
\setlength{\intextsep}{4.0pt plus 2.0pt minus 2.0pt}%
\setlength{\columnsep}{8pt}%

\usepackage{sidecap}
\usepackage{subcaption}
\usepackage[labelfont={bf,footnotesize}, textfont=footnotesize, skip=8pt plus 2pt minus 2pt]{caption}

\usepackage{mathtools} % should load amsmath
% \usepackage{amsmath}
\usepackage{amssymb}
\usepackage{amsfonts}
\usepackage{xspace}
\usepackage{changepage}


\usepackage{lineno}
\linenumbers

\usepackage{array}
\usepackage{booktabs}  % Nice tables!
\usepackage{multicol}
\usepackage{multirow}
\renewcommand{\arraystretch}{1.2}

\usepackage{siunitx}
\DeclareSIUnit{\molecule}{molecule}
\DeclareSIUnit{\molecules}{molecule}
\DeclareSIUnit\molar{\mole\per\cubic\deci\metre}
\DeclareSIUnit\Molar{\textsc{M}}
\sisetup{
    separate-uncertainty=true,
    detect-all, 
    range-phrase = --,
    list-units = single,
    range-units = single,
	group-digits = integer,
}

\usepackage[shortlabels]{enumitem}
\setlist{nosep} 

\usepackage[dvipsnames, table]{xcolor}

\usepackage[color=green!20]{todonotes}
\newcommand{\Red}[1]{{\color{red}#1}}
\newcommand{\Blue}[1]{{\color{blue}#1}}
\newcommand{\ctl}[1]{{\color{RubineRed}CTL: #1}}

\newcommand{\needref}[0]{\textcolor{red}{[REFS]}}
\newcommand{\thinsim}{{\raise.17ex\hbox{\(\scriptstyle\mathtt{\sim}\)}}}

%%%%%%%%%%%%%%%%%% ACRONYM SUPPORT %%%%%%%%%%%%%%%%%%%%%%%%%%%%%%%%%%%%%%%
% Learn more here: https://www.overleaf.com/learn/latex/glossaries
\usepackage[acronym, nonumberlist, nopostdot, nogroupskip, numberedsection=false]{glossaries}
% \setlength{\glsdescwidth}{0.8\textwidth}

\setacronymstyle{long-short}
\makeglossaries{}
\newacronym{md}{MD}{Molecular Dynamics}


\newcommand{\ep}{\epsilon}

\newcommand{\dd}{\mathrm{d}}
\newcommand{\ii}{\mathrm{i}}
\newcommand{\ee}{\mathrm{e}}

\newcommand{\la}{\langle}
\newcommand{\ra}{\rangle}
\newcommand{\defn}{\stackrel{\text{def}}{=}}

\DeclareSIUnit[]{\osm}{Osm}

\begin{document}
\begin{centering}
    \textbf{\Large Notes on axon pearling simulations}\\[3mm]
\end{centering}

\section{Equilibrium conditions for tubes}
Following the approach outlined by Refs.~\cite{DerenyiEtAl2002,ShurerEtAl2019}, we first establish the general conditions for pulling a tube from a reservoir of membrane.
The general form of the energy of a membrane with an external force \(f\) acting over length \(L\) is,
\begin{equation}
    F = \int_A \left[\kappa(H - c_0)^2 + \sigma \right]dA - pV - fL.
\end{equation}
Where \(H\) is the mean curvature give by \((c_1 + c_2)/2\)
In the special case of a cylinder which has \(c_1 = 1/R\) and \(c_2 = 0\), assuming that \(p=0\), and \(f\neq 0\),
\begin{equation}
    F = \left[ \kappa\left(\frac{1}{2R} - c_0\right)^2 + \sigma\right] 2\pi RL - fL.
\end{equation}

At equilibrium the energies are minimized with respect to the radius and length of the tube such that
\begin{equation}
    \frac{\partial F}{\partial R} = 0,\quad \frac{F}{\partial L} =0.
\end{equation}
Considering first the minimization with respect to the radius,
\begin{align*}
    0 = \frac{\partial F}{\partial R} & = 2\pi L\left[\frac{-\kappa}{4R^2} + \kappa c_0^2 + \sigma\right] \\
                                      & = - \frac{\kappa}{4R^2} + \kappa c_0^2 + \sigma
\end{align*}
Thus,
\begin{equation}
    R = \frac{1}{2\sqrt{c_0^2 + \sigma/\kappa}},
\end{equation}
relates the radius to the spontaneous curvature, tension, and bending rigidity; moreover,
\begin{equation}
    c_0 = \sqrt{1/4R^2 - \sigma/\kappa}.
\end{equation}

Assuming an initial geometry of a cylinder with radius \(R = \SI{0.05}{\micro\meter}\) and given values of tension and bending rigidity, we can obtain the homeogenous spontaneous curvature to sustain the tube.

\section{Osmotic pressure relationship and computing pressure force}

The osmolarity of the solution, \(O\), is given by
\begin{equation}
    O = \sum_i \varphi_i k_i C_i,
\end{equation}
where \(\varphi\) is the osmotic (activity) coefficient, \(k\) is the number of dissociated ions, and \(C\) is the concentration.

The osmotic pressure is given by
\begin{equation}
    P = RTO  = RT \sum_i  \varphi_i k_i C_i
\end{equation}

Simplifying by homogenizing the contributions of each ionic species, the osmotic pressure energy is given by,
\begin{equation}
    E_p = \int_{\bar{V}}^{V} \Delta P(\tilde{V}; \vec{r})d \tilde{V} = \int_{\bar{V}}^{V} kRT \left(\frac{n}{V} - \bar{c}\right) dV = kRTn[\bar{c}V/n - \ln(\bar{c}V/n) -1],
\end{equation}
where \(RT\) is the ideal gas constant and temperature, \(n\) is the amount of enclosed solute within the geometry, \(\bar{c}\) is the concentration of the external solvent, and \(V\) is the current volume.

To define the initial molar amount of enclosed solute, we assume that the interior of the cell is at the conventional \(c_{t=0} = \SI{300}{\milli\osm}\).
The amount of solute is given by the product of concentration and initial volume, \(n = c_{t=0}V_{t=0}\). 
Since the initial geometry, a cylinder with radius \SI{0.05}{\micro\meter}, is somewhat arbitrary, we scale the initial volume by a factor of three to set the effective target volume,
\begin{equation}
    n = c_{t=0}\cdot3V_{t=0}.
\end{equation}

An effect of this scaling is that the initial volume is much different than the target volume. 
The contributions from high pressure can lead to numerical instabilities. 
We assume that since we are modeling only a small section of axon with the majority of the cell volume not represented, that the bulk of the cell also responds to osmotic pressure.
To capture the contribution implicitly, we scale the energy by a reservoir volume \(V_r\) nondimensionalized by a unit volume,
\begin{equation}
    E_p =\frac{kRTn}{V_r}[\bar{c}V/n - \ln(\bar{c}V/n) -1].
\end{equation}
For our intuition, we note that at nearly isoosmotic conditions,
\begin{equation}
    \lim_{V\to\bar{V}} kRTn[\bar{c}V/n - \ln(\bar{c}V/n) -1] \approx \frac{1}{2}K_v \frac{(V - \bar{V})^2}{\bar{V}^2} + \mathcal{O}(V^3).
\end{equation}
we can define a constant \(K_v = kRTn\) which accounts for the constant parameters.
Thus the reservoir volume is serving to temper the effective bulk modulus.


\printbibliography
\end{document}
