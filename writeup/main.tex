\documentclass[11pt]{article}
%---------------------------------
% Set page dimensions
%---------------------------------
\usepackage[margin=1in]{geometry}
% \setlength{\parindent}{0em}
\setlength{\parskip}{0em}
%=================================

% \usepackage[T1]{fontenc}
% \usepackage[utf8]{inputenc}
% \usepackage{times}
% \usepackage{helvet}

%---------------------------------
% Font options
%---------------------------------
\usepackage{fontspec}
\setmainfont{Arial}

\usepackage[style=science, doi=true, backend=biber]{biblatex}
\setlength{\bibitemsep}{1pt}
% \renewcommand\bibfont{\normalfont\scriptsize}
\addbibresource{ref.bib}

\usepackage{graphicx}
\usepackage{float}
\PassOptionsToPackage{hyphens}{url}\usepackage[colorlinks=false]{hyperref}
\usepackage{bookmark}
\usepackage[all]{hypcap}
\usepackage[version=4]{mhchem}
\usepackage[capitalise]{cleveref}

\usepackage{wrapfig}
\setlength{\intextsep}{4.0pt plus 2.0pt minus 2.0pt}%
\setlength{\columnsep}{8pt}%

\usepackage{sidecap}
\usepackage{subcaption}
\usepackage[labelfont={bf,footnotesize}, textfont=footnotesize, skip=8pt plus 2pt minus 2pt]{caption}

\usepackage{mathtools} % should load amsmath
% \usepackage{amsmath}
\usepackage{amssymb}
\usepackage{amsfonts}
\usepackage{xspace}
\usepackage{changepage}


\usepackage{lineno}
\linenumbers

\usepackage{array}
\usepackage{booktabs}  % Nice tables!
\usepackage{multicol}
\usepackage{multirow}
\renewcommand{\arraystretch}{1.2}

\usepackage{siunitx}
\DeclareSIUnit{\molecule}{molecule}
\DeclareSIUnit{\molecules}{molecule}
\DeclareSIUnit\molar{\mole\per\cubic\deci\metre}
\DeclareSIUnit\Molar{\textsc{M}}
\sisetup{
    separate-uncertainty=true,
    detect-all, 
    range-phrase = --,
    list-units = single,
    range-units = single,
	group-digits = integer,
}

\usepackage[shortlabels]{enumitem}
\setlist{nosep} 

\usepackage[dvipsnames, table]{xcolor}

\usepackage[color=green!20]{todonotes}
\newcommand{\Red}[1]{{\color{red}#1}}
\newcommand{\Blue}[1]{{\color{blue}#1}}
\newcommand{\ctl}[1]{{\color{RubineRed}CTL: #1}}

\newcommand{\needref}[0]{\textcolor{red}{[REFS]}}
\newcommand{\thinsim}{{\raise.17ex\hbox{\(\scriptstyle\mathtt{\sim}\)}}}

%%%%%%%%%%%%%%%%%% ACRONYM SUPPORT %%%%%%%%%%%%%%%%%%%%%%%%%%%%%%%%%%%%%%%
% Learn more here: https://www.overleaf.com/learn/latex/glossaries
\usepackage[acronym, nonumberlist, nopostdot, nogroupskip, numberedsection=false]{glossaries}
% \setlength{\glsdescwidth}{0.8\textwidth}

\setacronymstyle{long-short}
\makeglossaries{}
\newacronym{md}{MD}{Molecular Dynamics}


\newcommand{\ep}{\epsilon}

\newcommand{\dd}{\mathrm{d}}
\newcommand{\ii}{\mathrm{i}}
\newcommand{\ee}{\mathrm{e}}

\newcommand{\la}{\langle}
\newcommand{\ra}{\rangle}
\newcommand{\defn}{\stackrel{\text{def}}{=}}

\DeclareSIUnit[]{\osm}{Osm}

\begin{document}
\begin{centering}
    \textbf{\Large Notes on axon pearling}\\[3mm]
    \textbf{Christopher T. Lee, Padmini Rangamani}\\[1mm]
\end{centering}

\section{Equilibrium tube radius}
Following the approach outlined by Refs.~\cite{DerenyiEtAl2002,ShurerEtAl2019}.
The general form of the energy of a membrane with an external force \(f\) acting over length \(L\) is,
\begin{equation}
    F = \int_A \left[\kappa(H - c_0)^2 \sigma \right]dA - pV - fL.
\end{equation}
Where \(H\) is the mean curvature give by \((c_1 + c_2)/2\)
In the special case of a cylinder which has \(c_1 = 1/R\) and \(c_2 = 0\), assuming that \(p=0\), and \(f\neq 0\),
\begin{equation}
    F = \left[ \kappa\left(\frac{1}{2R} - c_0\right)^2 + \sigma\right] 2\pi RL - fL.
\end{equation}

At equilibrium the energies are minimized with respect to the radius and length of the tube such that
\begin{equation}
    \frac{\partial F}{\partial R} = 0,\quad \frac{F}{\partial L} =0.
\end{equation}
Considering first the minimization with respect to the radius,
\begin{align*}
    0 = \frac{\partial F}{\partial R} & = 2\pi L\left[\frac{-\kappa}{4R^2} + \kappa c_0^2 + \sigma\right] \\
                                      & = - \frac{\kappa}{4R^2} + \kappa c_0^2 + \sigma
\end{align*}
Thus,
\begin{equation}
    \boxed{R = \frac{1}{2\sqrt{c_0^2 + \sigma/\kappa}}}
\end{equation}

Rearranging,
\begin{equation}
    c_0 = \sqrt{1/4R^2 - \sigma/\kappa}.
\end{equation}

For \(R = \SI{0.05}{\micro\meter}\) the limit of \(\sigma/\kappa\) is \(\approx \SI{100}{\per\micro\meter\squared}\).

\section{Osmotic pressure relationship}

The osmolarity of the solution, \(O\), is given by
\begin{equation}
    O = \sum_i \varphi_i n_i C_i,
\end{equation}
where \(\varphi\) is the osmotic (activity) coefficient, \(n\) is the number of dissociated ions, and \(C\) is the concentration.

The osmotic pressure is given by
\begin{equation}
    P = RTO  = RT \sum_i  \varphi_i n_i C_i
\end{equation}

We assume that the interior of the cell is at the conventional \SI{300}{\milli\osm}.

The osmotic pressure energy
\begin{equation}
    E_p = \int_{\bar{V}}^{V} \Delta P(\tilde{V}; \vec{r})d \tilde{V} = \int_{\bar{V}}^{V} iRT \left(\frac{n}{V} - \bar{c}\right) dV = iRTn[\bar{c}V/n - \ln(\bar{c}V/n) -1]
\end{equation}

At nearly isoosmotic conditions, 
\begin{equation}
    \lim_{V\to\bar{V}} iRTn[\bar{c}V/n - \ln(\bar{c}V/n) -1] \approx \frac{1}{2}K_v \frac{(V - \bar{V})^2}{\bar{V}} + \mathcal{O}(V^3).
\end{equation}
We define a constant \(K_v = iRTn\) accounting for the constant parameters.


\printbibliography
\end{document}
